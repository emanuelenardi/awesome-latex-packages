% NOTE for a set of predefined color names, like LemonChiffon1
\PassOptionsToPackage{
	x11names
}{xcolor}

\documentclass{article}

\usepackage{
	,listings
	,algorithm2e
	,textcomp
	,xcolor
}

% NOTE allow "No Starch Press"-like custom line numbers (essentially, bulleted line
% numbers for only those lines the author will address)
\usepackage{
	pifont,
	ifthen
}

\newcommand{\numold}[1]{\oldstylenums{#1}}
\newcounter{lstNoteCounter}
\newcommand*{\lnote}{%
	\stepcounter{lstNoteCounter}\vbox{\llap{{\lnnum{\thelstNoteCounter}}\hskip 1em}}%
}

\newcommand{\lnnum}[1]{% Print pifont circled number for line label
	\ifcase#1%
	% nothing for 0
	\or\ding{202}% 1
	\or\ding{203}% 2
	\or\ding{204}% 3
	\or\ding{205}% 4
	\or\ding{206}% 5
	\or\ding{207}% 6
	\or\ding{208}% 7
	\or\ding{209}% 8
	\or\ding{210}% 9
	\or\ding{211}% 10
	\else{NUM TOO HIGH}%
	\fi%
}

\lstnewenvironment{csource2}[1][]
{
    \setcounter{lstNoteCounter}{0}
    \lstset{
		basicstyle = \ttfamily,
		language = C,
		numberstyle = \numold,
		numbers = right,
        frame = lines,
		framexleftmargin = 0.5em,
		framexrightmargin = 0.5em,
        backgroundcolor = \color{LemonChiffon1},
		showstringspaces = false,
        escapeinside = {(*@}{@*)},
		%
		#1
	}
}
{}

% arara: pdflatex
% arara: latexmk: { clean: partial }
\begin{document}

\section{Hello, world!}
The following program \texttt{hello.c} simply prints ``Hello, world!'':

\begin{csource2}
(*@\lnote@*)#include <stdio.h>

/* This is a comment. */
(*@\lnote@*)int main() {
(*@\lnote@*)    printf("Hello, world!\n");
(*@\lnote@*)    return 0;
}
\end{csource2}

\noindent
We first include the \texttt{stdio.h} header file \lnnum{1}. We then declare
the \texttt{main} function \lnnum{2}. We then print ``Hello, world!'' to
standard output (a.k.a., \textit{STDOUT}) \lnnum{3}. Finally, we return value
0 to let the caller of this program know that we exited safely without any
errors \lnnum{4}.


\end{document}
