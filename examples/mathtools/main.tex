\documentclass{article}

\usepackage{
	ams-all,
	mathtools,
	enumitem,
	ulem,
}

\newtheorem{corollario}{Corollario}
\newtheorem{definizione}{Definizione}
\newtheorem{esempio}{Esempio}
\newtheorem{oss}{Osservazione}
\newtheorem{proposizione}{Proposizione}
\newtheorem{theorem}{Teorema}
\newtheorem*{lemma}{Lemma}

% NOTE: comando definito dal package "mathtools"
\DeclarePairedDelimiter\norm{\lVert}{\rVert}
\DeclarePairedDelimiter\abs{\lvert}{\rvert}
\DeclarePairedDelimiter\Abs{\bigg\lvert}{\bigg\rvert}
\DeclarePairedDelimiter\Bracket{\lbrack}{\rbrack}

% arara: pdflatex: { synctex: yes }
% arara: latexmk: { clean: partial }
\begin{document}

\begin{equation}
	\frac{d}{dx} \tan(x) = \frac{d}{dx} \frac{\sin(x)}{\cos(x)} = \frac{\sin(x)' \cdot \cos(x) + \sin(x) \cdot \cos(x)'}{{\sin(x)}^{2} }
\end{equation}

\begin{theorem}
Sia \(f(a, b)\) continua, sia \(x_0 \in (a, b)\). %
Supp. \(f\) defivabile in ogni \(x \neq x_0\).

Allora:
\begin{itemize}
	\item se
	\item se
\end{itemize}
\end{theorem}

\end{document}
