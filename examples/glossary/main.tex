\documentclass{article}

\usepackage[utf8]{inputenc}
\usepackage[T1]{fontenc}
\usepackage[italian]{babel}

\usepackage{xcolor}

% NOTE glossario
\usepackage[toc, acronym, numberedsection]{glossaries-extra}
\makeglossaries
% NOTE termini

\newglossaryentry{latex}{
    name=latex,
    description={Is a mark up language specially suited for scientific documents}
}

\newglossaryentry{maths}{
    name=mathematics,
    description={Mathematics is what mathematicians do}
}

\newglossaryentry{formula}{
    name=formula,
    description={A mathematical expression}
}

% NOTE acronimi

\newacronym{gcd}{GCD}{Greatest Common Divisor}

\newacronym{lcm}{LCM}{Least Common Multiple}


% NOTE stili disponibili
% \setglossarystyle{list} % default
\setglossarystyle{altlist}
% \setglossarystyle{listgroup}
% \setglossarystyle{listhypergroup}
% \setglossarystyle{long}

\title{Utilizzo del glossario}
\author{Emanuele Nardi}
\date{\today}

% NOTE ciclo di compilazione
% arara: pdflatex
% arara: makeglossaries
% arara: pdflatex
% arara: clean: { extensions: [acn, acr, alg, glg, glo, gls, ist] }
% arara: latexmk: { clean: partial }

% NOTE ultima modifica 2019-03-09
\begin{document}
\maketitle

% NOTE cambio intestazione "indice"
\renewcommand{\contentsname}{Indice dei contenuti}
% NOTE indice dei contenuti
\tableofcontents

\section{Esempio con termini}

The \verb|\Gls{latex}| \textcolor{blue}{\Gls{latex}} typesetting markup language is specially suitable for documents that include \verb|\gls{maths}| \textcolor{blue}{\gls{maths}}. \Glspl{formula} are rendered properly an easily once one gets used to the commands.

\section{Esempio con acronimi}

Given a set of numbers, there are elementary methods to compute its \verb|\acrlong{gcd}| \textcolor{blue}{\acrlong{gcd}}, which is abbreviated \verb|\acrshort{gcd}| \textcolor{blue}{\acrshort{gcd}}.
This process is similar to that used for the \verb|\acrfull{lcm}| \textcolor{blue}{\acrfull{lcm}}.

% NOTE riempimento della pagina
\vfill

% NOTE stampa il glossario
\printglossary[title=Lista dei termini, toctitle=Glossario]

% NOTE stampa solo gli acronimi
\printglossary[type=\acronymtype, title=Lista degli acronimi, toctitle=Acronimi]

\end{document}
