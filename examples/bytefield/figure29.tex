\documentclass[border={5pt 5pt 5pt 10pt}]{standalone}

\usepackage{bytefield}



% arara: pdflatex: { synctex: no }
% arara: latexmk: { clean: partial }
\begin{document}
\begin{bytefield}[bitheight=2.5\baselineskip]{32}

	\bitheader{0,7,8,15,16,23,24,31} \\
	\begin{rightwordgroup}{\parbox{6em}{%
		\raggedright These words were taken verbatim from the TCP header definition (RFC~793).}}

		\bitbox{4}{Data offset} &
		\bitbox{6}{Reserved} &
		\bitbox{1}{\tiny U\\R\\G} &
		\bitbox{1}{\tiny A\\C\\K} &
		\bitbox{1}{\tiny P\\S\\H} &
		\bitbox{1}{\tiny R\\S\\T} &
		\bitbox{1}{\tiny S\\Y\\N} &
		\bitbox{1}{\tiny F\\I\\N} &
		\bitbox{16}{Window} \\

		\bitbox{16}{Checksum} &
		\bitbox{16}{Urgent pointer}
	\end{rightwordgroup} \\

	\wordbox[lrt]{1}{Data octets} \\
	\skippedwords \\
	\wordbox[lrb]{1}{} \\

	\begin{leftwordgroup}{\parbox{6em}{%
		\raggedright Note that we can display, for example, a misaligned 64-bit value with clever use of the optional argument to \texttt{\string\wordbox} and \texttt{\string\bitbox}.}}

	\bitbox{8}{Source} & \bitbox{8}{Destination} & \bitbox[lrt]{16}{} \\
	\wordbox[lr]{1}{Timestamp} \\

	\begin{rightwordgroup}{\parbox{6em}{%
		\raggedright Why two Length fields? No particular reason.}}

	\bitbox[lrb]{16}{} & \bitbox{16}{Length} \end{leftwordgroup} \\
	\bitbox{6}{Key} & \bitbox{6}{Value} & \bitbox{4}{Unused} & \bitbox{16}{Length} \end{rightwordgroup} \\

	\wordbox{1}{Total number of 16-bit data words that follow this header word, excluding the subsequent checksum-type value} \\

	\bitbox{16}{Data~1} & \bitbox{16}{Data~2} \\
	\bitbox{16}{Data~3} & \bitbox{16}{Data~4} \\
	\bitbox[]{16}{$\vdots$ \\[1ex]} & \bitbox[]{16}{$\vdots$ \\[1ex]} \\
	\bitbox{16}{Data~$N-1$} & \bitbox{16}{Data~$N$} \\
	\bitbox{20}{\[ \mbox{A5A5}_{\mbox{\scriptsize H}} \oplus \left(\sum_{i=1}^N \mbox{Data}_i \right) \bmod 2^{20} \]} & \bitboxes*{1}{000010 000110} \\
	\wordbox{2}{64-bit random number}
\end{bytefield}
\end{document}
